\documentclass[11pt]{article}

\usepackage[utf8]{inputenc}
\usepackage[english,greek]{babel}
\babeltags{en = english}
\babeltags{gr = greek}
\usepackage{fullpage,enumitem,amsmath,amssymb,graphicx}
\usepackage{alphabeta}
\usepackage{blindtext}
\usepackage{tikz}
\usepackage{graphicx}
\usepackage{tikz-3dplot,pgfplots}
\usepackage{wrapfig}
\usepackage[]{hyperref}
\usepackage{verbatim}

\begin{document}

\begin{center}
{\LARGE Προχωρημένα Θέματα Βάσεων Δεδομένων}

\begin{tabular}{ll}
Ονοματεπώνυμο: & Μπάρμπα Παναγιώτα-Νικολέττα \\
ΑΜ : & 03118604 \\
Εξάμηνο : & 11ο \\
Ομάδα : 41 \\
\texten{Github} : & \href{https://github.com/NicoleMp2/advDatabases}{\texten{Github Link}} \\
\end{tabular}
\end{center}



\section*{Ζητούμενο 1}

\par Η εγκατάσταση και διαμόρφωση της πλατφόρμας εκτέλεσης Apache Spark ώστε να εκτελείται πάνω από το διαχειριστή πόρων του Apache Hadoop, YARN, έγινε σε 2 εικονικά μηχανήματα σε τοπικό μηχάνημα (δεν χρησιμοποιήθηκε το \texten{cloud service okeanos}). Η διαμόρφωση των εργαλείων που χρησιμοποιήθηκαν περιγράφεται στο \texten{README} αρχείο του \texten{Github} αποθετηρίου.
\par Οι \texten{web} διεπαφές των \texten{Apache Spark} και \texten{Apache Hadoop} είναι προσβάσιμες από τους παρακάτω συνδέσμους:
\begin{itemize}
  \item \texten{Apache Spark} : \href{http://192.168.64.9:8080/}{\texten{http://192.168.64.9:8080/}}
  \item \texten{Apache Hadoop} : \href{http://192.168.64.9:9870/}{\texten{http://192.168.64.9:9870/}}
  \item \texten{Apache Hadoop YARN} : \href{http://192.168.64.9:8088/}{\texten{http://192.168.64.9:8088/}}
\end{itemize}

\section*{Ζητούμενο 2}
Αυτό όπως και τα επόμενα ζητούμενα υλοποιήθηκαν με χρήση του \texten{PySpark} και της γλώσσας προγραμματισμού \texten{Python3}.
\par Δημιουργήθηκε ένα \texten{DataFrame} από το βασικό σύνολο δεδομένων και διατηρώντας τα ονόματα των στηλών, προσαρμόστηκαν οι τύποι ορισμένων στηλών ως εξής:
\begin{itemize}
  \item \texten{Date Rptd} : \texten{string} $\rightarrow$ \texten{date}
  \item \texten{DATE OCC} : \texten{string} $\rightarrow$ \texten{date}
  \item \texten{Vict Age} : \texten{string} $\rightarrow$ \texten{integer}
  \item \texten{LAT} : \texten{double} $\rightarrow$ \texten{double}
  \item \texten{LON} : \texten{double} $\rightarrow$ \texten{double}
\end{itemize}
\par Επίσης, στο αρχειο \texten{IncomeData2015.csv} η στήλη  \texten{"Estimated Median Income"} έχει τύπο \texten{string} της μορφής: "\$\texten{number}", οπότε αφαιρέθηκε το '\$' και έγινε μετατροπή σε \texten{integer}.
\par Τέλος, ενώθηκαν τα \texten{DataFrame} που περιέχουν τα δεδομένα καταγραφής εγκλημάτων για το \texten{Los Angeles} από το 2010 μέχρι το 2019 και από το 2020 μέχρι σήμερα, τα δεδομένα με \texten{reverse geocoding} πληροφορία και τα δεδομένα σχετικά με το μέσο εισόδημα ανά νοικοκυριό και ταχυδρομικό κώδικα δημιουργώντας ένα νέο \texten{DataFrame}, το οποίο αποθηκέυτηκε, με την εξής μορφή:
\texten{\verbatiminput{outputs/ConfigData.txt}}

\section*{Ζητούμενο 3}


\section*{Ζητούμενο 4}



\end{document}
