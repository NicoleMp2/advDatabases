\documentclass[11pt]{article}

\usepackage[utf8]{inputenc}
\usepackage[english,greek]{babel}
\babeltags{en = english}
\babeltags{gr = greek}
\usepackage{fullpage,enumitem,amsmath,amssymb,graphicx}
\usepackage{alphabeta}
\usepackage{blindtext}
\usepackage{tikz}
\usepackage{graphicx}
\usepackage{tikz-3dplot,pgfplots}
\usepackage{wrapfig}
\usepackage[]{hyperref}
\begin{document}

\begin{center}
{\LARGE Προχωρημένα Θέματα Βάσεων Δεδομένων}

\begin{tabular}{rl}
Ονοματεπώνυμο: & Μπάρμπα Παναγιώτα-Νικολέττα \\
ΑΜ : & 03118604 \\
Εξάμηνο : & 11ο \\
Ομάδα : 41 \\
\texten{Github} : & \href{https://github.com/NicoleMp2/advDatabases}{\texten{Github Link}} \\
\end{tabular}
\end{center}



\section*{Ζητούμενο 1}

\par Η εγκατάσταση και διαμόρφωση της πλατφόρμας εκτέλεσης Apache Spark ώστε να εκτελείται πάνω από το διαχειριστή πόρων του Apache Hadoop, YARN, έγινε σε 2 εικονικά μηχανήματα σε τοπικό μηχάνημα (δεν χρησιμοποιήθηκε το \texten{cloud service okeanos}). 
\par Οι \texten{web} διεπαφές των \texten{Apache Spark} και \texten{Apache Hadoop} είναι προσβάσιμες από τους παρακάτω συνδέσμους:
\begin{itemize}
  \item \texten{Apache Spark} : \href{http://192.168.64.9:8080/}{\texten{http://192.168.64.9:8080/}}
  \item \texten{Apache Hadoop} : \href{http://192.168.64.9:9870/}{\texten{http://192.168.64.9:9870/}}
  \item \texten{Apache Hadoop YARN} : \href{http://192.168.64.9:8088/}{\texten{http://192.168.64.9:8088/}}
\end{itemize}


\section*{Ζητούμενο 3}


\section*{Ζητούμενο 4}



\section*{Ζητούμενο 6}


% \begin{figure}[h]
%   \includegraphics[width=10cm,height=6cm]{Screenshot 2023-12-29 at 20.14.34.png}
%   \centering
% \end{figure}

\end{document}
